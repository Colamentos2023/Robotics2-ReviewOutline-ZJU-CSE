\documentclass[a4paper,11pt]{article}
\usepackage{ctex}
\usepackage{geometry}
\usepackage{float}
\geometry{a4paper, margin=1in}
\usepackage{tocloft}
\usepackage{titling}
\usepackage{amsmath, amssymb, amsfonts}
\usepackage{graphicx}
\usepackage{booktabs}
\usepackage{hyperref}
\usepackage{xeCJK}
\usepackage{xcolor}
\usepackage{enumitem}
\usepackage{indentfirst}
\usepackage{fancyhdr}
\usepackage{tikz}
\usepackage{array}
\usepackage{colortbl}
\usepackage{caption}
\usepackage{subcaption}
\usepackage{subfiles}
\usepackage{threeparttable}
\usepackage{makecell} 
\usepackage{array} 

\DeclareGraphicsExtensions{.pdf,.png,.jpg,.jpeg}
\graphicspath{{./figs/}{./out/}}

% 页眉页脚
\pagestyle{fancy}
\fancyhf{}
\fancyhead[C]{机器人学II 复习提纲}
\fancyfoot[C]{\thepage}
\setlength{\headheight}{15pt}

% 标题
\title{机器人学II 复习提纲 \\ \vspace{0.5cm} \large }
\author{collamentos@collamentos-JiguangPro-Series-GM5AR0O}
\date{2025年9月}

\begin{document}
\begin{titlepage}
    \centering
    \vspace*{3cm}
    {\Huge \textbf{机器人学II 复习提纲} \par}
    \vspace{0.5cm}
    {\Large 授课教师:李硕、李亮\par}
    \vspace{3cm}
    {\Large Colamentos \par}
    {\Large 2025-2026秋冬学期 \par}
\end{titlepage}

\section*{前言}

本学期首次熊蓉老师退出该课程, 改为李硕老师和李亮老师讲解。李硕老师的授课风格和内容与前几届有所不同,且考试风格可能也会有所变动。因此在面向往年历年卷复习可能出现偏差,且该课程实际上主要仍以背诵、记忆内容为主的情况下,我主要根据ppt内容整理了本提纲,主要面向23级机器人工程专业同学应对课程《机器人学II》考试时参考使用,并希望为后几届同学也能提供参考。

笔者上课基本也没有听,整理也十分仓促,疏漏或不足较多,恳请批评指正,也欢迎同学们以及学弟学妹们直接用本tex代码继续在未来根据实际情况做出补充和修正!

祝大家学习顺利!

\begin{flushright}
Colamentos\\
2025年10月28日于玉泉31舍
\end{flushright}
\newpage
\tableofcontents
\thispagestyle{empty}
\newpage

%=============================
%=============================
\subfile{sections/path_planning.tex}
\subfile{sections/obstacle_avoidance_planning.tex}
\subfile{sections/trajectory_planning}
\subfile{sections/navigation_planning_fusion_optimization.tex}
\end{document}
