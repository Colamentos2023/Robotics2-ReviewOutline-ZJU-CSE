\documentclass[../main.tex]{subfiles}
\begin{document}
\section{定位问题}
\subsection{定位问题的基本概念}
\subsection{基于外部设备感知的定位}
\begin{enumerate}
    \item \textbf{全球定位系统GPS}
    \item \textbf{全局视觉观测定位}
\end{enumerate}
\subsection{基于本体感知的定位}
\begin{enumerate}
    \item \textbf{基于空间标识的定位}
        \begin{itemize}
            \item \textbf{基于环境人工标识的定位}
            \item \textbf{基于环境自然标识的定位}
        \end{itemize}
    \item \textbf{基于位置识别的定位}
\end{enumerate}
\begin{itemize}
    \item \textbf{优点}:
        \begin{itemize}
            \item 
        \end{itemize}
    \item \textbf{缺点}:
        \begin{itemize}
            \item 
        \end{itemize}
\end{itemize}
\subsection{控制感知信息融合的的自定位}
\begin{enumerate}
    \item \textbf{基本概念}
    \item \textbf{概率描述}
\end{enumerate}
\subsection{马尔可夫定位中的运动模型}
\begin{enumerate}
    \item \textbf{里程计运动模型}
    \item \textbf{速度运动模型}
\end{enumerate}
\subsection{马尔可夫定位中的观测模型}
\begin{enumerate}
    \item \textbf{特征传感器模型和观测模型}
    \item \textbf{基于物理建模的激光传感器模型}
    \item \textbf{基于Likelihood Field的激光束模}
\end{enumerate}
\end{document}